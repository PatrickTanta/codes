Eu agrede�o � minha fam�lia pelo apoio e pelas oportunidades que tive.

Agrade�o ao Prof. Marcelo J.S. de Lemos pela orienta��o e pela ajuda com os recursos computacionais. Para mim, foi um privil�gio trabalhar sob orienta��o de um pesquisador de sucesso exemplar.

Agrade�o � Sygma Motors, nominalmente a Marcos Langeani, onde iniciei minha vida profissional e a quem sou grato pelas oportunidades e pelo aprendizado oferecidos.

Agrade�o ao Prof. Francisco Jos� de Souza pelo aprendizado, pelo incentivo ao estudo da mec�nica dos fluidos e pelo exemplo profissional.

Agrade�o aos ex-colegas Carla Fernandes, Clayton Zabeu, Edilson Viana, Felipe Tannus Dorea, Gustavo Hindi, Jos� Eduardo de Oliveira, Marcelo Andreotti e Oswaldo Fran�a Jr pelo companheirismo e pelo aprendizado.

Agrade�o ao Prof. Fl�vio Luiz Cardoso Ribeiro, procurador para os tr�mites burocr�ticos do ITA, colega no curso de gradua��o em engenharia aeron�utica e um dos meus melhores amigos, pelo apoio � conclus�o dos estudos do mestrado.

Agrade�o � Blandina Oliveira pela ajuda com a impress�o e com a distribui��o das c�pias da disserta��o.

Agrade�o �s comunidades de c�digo aberto que mant�m softwares de qualidade disponibilizados sem custo pelo simples esp�rito de colabora��o.

Este trabalho come�ou na quase sub-tropical S�o Jos� dos Campos e foi finalizado em Salvador, no cora��o do Brasil.
