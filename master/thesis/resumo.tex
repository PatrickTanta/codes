
Este trabalho consistiu na simula��o num�rica de um spray: um escoamento bif�sico composto por uma fase gasosa e uma fase l�quida. A fase gasosa � tratada como um meio cont�nuo, e a fase l�quida � tratada com um conjunto de gotas dispersas na fase gasosa. Foi aplicada a aproxima��o assint�tica de baixo n�mero de Mach para a fase gasosa com o objetivo de modelar varia��es da densidade causadas por gradientes de temperatura sem o envolvimento de complica��es existentes na formula��o de um escoamento compress�vel. Os efeitos da turbul�ncia na fase gasosa foram modelados utilizando o conceito de viscosidade turbulenta governada por duas equa��es diferenciais que comp�e o assim denominado modelo \textit{k-epsilon}. A fase l�quida dispersa foi modelada como part�culas-pontos, �s quais s�o atribu�das propriedades termodin�micas e modelos 0-d para c�lculos de transfer�ncia de momento, energia e massa. A fase gasosa foi discretizada segundo o m�todo de volumes finitos. Os resultados num�ricos foram comparados com medi��es experimentais e indicaram que a metodologia utilizada descreve razoavelmente bem o spray. Foram verificadas, por�m, subestimativas da velocidade e da taxa de evapora��o das gotas com rela��o �s medi��es experimentais.
